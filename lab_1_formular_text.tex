%\documentclass[12pt,a4paper]{article}
%\usepackage{tech_doc_rus}
%\makenomenclature %Закомментировать, если не нужен

%\begin{document}

%\tableofcontents

%\newpage
%\addcontentsline{toc}{section}{\tocsecindent{Обозначения и сокращения}}
%\renewcommand{\nomname}{Обозначения и сокращения}
%\printnomenclature
%\newpage

%\section*{Аннотация}


\makenomenclature

\nomenclature{ПО}{Программное обеспечение --- Совокупность программ, используемых для выполнения определённых задач на вычислительной системе}
\nomenclature{ESXi}{VMware ESXi --- Гипервизор, обеспечивающий виртуализацию на уровне ядра операционной системы}
\nomenclature{vCenter}{vCenter Server --- Центральный сервер управления инфраструктурой виртуализации от VMware}
\nomenclature{vSwitch}{Virtual Switch --- Виртуальный коммутатор, обеспечивающий сетевое взаимодействие между виртуальными машинами и физической сетью}
\nomenclature{VLAN ID}{Virtual Local Area Network Identifier --- Идентификатор виртуальной локальной сети, используемый для разделения трафика внутри одной физической сети}
\nomenclature{MTU}{Maximum Transmission Unit --- Максимальный размер передаваемого сетевого пакета, измеряется в байтах}
\nomenclature{PowerCLI}{VMware PowerCLI --- Набор PowerShell-модулей для управления инфраструктурой VMware через скрипты}
\nomenclature{PowerShell 7}{PowerShell Core 7 --- Кроссплатформенная версия PowerShell, поддерживающая современные функции и .NET Core}
\nomenclature{Port Group}{Порт-группа --- Логическая группа портов на виртуальном коммутаторе, имеющая общие параметры VLAN и безопасности}
\nomenclature{Maintenance Mode}{Режим обслуживания --- Состояние хоста ESXi, при котором он временно исключается из активной эксплуатации для проведения работ}
\nomenclature{TLS}{Transport Layer Security --- Протокол шифрования, обеспечивающий безопасную передачу данных по сети}
\nomenclature{HTTPS}{HyperText Transfer Protocol Secure --- Безопасный протокол передачи данных поверх TLS}
\nomenclature{CI/CD}{Continuous Integration / Continuous Delivery --- Методологии автоматизации разработки и доставки программного обеспечения}
\nomenclature{Git}{Git --- Система контроля версий, используемая для отслеживания изменений в коде программы}

\section{Общие указания}

Формуляр содержит сведения, необходимые для эксплуатации, хранения и сопровождения программного изделия «Программа конфигурации хостов виртуализации» (шифр ЛАБ 1). 

Формуляр составлен в соответствии с требованиями ГОСТ 19.402–78 и подлежит применению при приёмке, вводе в эксплуатацию, сопровождении и хранении программного изделия.

Документ подлежит обязательному учёту и актуализации при внесении изменений в программное изделие.

\newpage
\section{Общие сведения}

\begin{itemize}
  \item \textbf{Наименование программного изделия:} Программа конфигурации хостов виртуализации
  \item \textbf{Обозначение программного изделия:} ЛАБ 1.ПК-ПУПС
  \item \textbf{Предприятие-изготовитель:} ФГБОУ ВО МГТУ ГА, кафедра ОРТЗИ
  \item \textbf{Номер программного изделия предприятия:} 01132212.62.01.12-01 81 03
  \item \textbf{Версия программного изделия:} 1.0
  \item \textbf{Год выпуска:} 2025
\end{itemize}

Программное изделие предназначено для автоматизации процесса настройки сетевой инфраструктуры хостов VMware ESXi при их добавлении или восстановлении в рамках частного облачного окружения.
\newpage
\section{Основные характеристики}

\begin{itemize}
  \item Среда выполнения: PowerShell 7.x
  \item Зависимости: VMware PowerCLI 12.0 или выше
  \item Поддерживаемые версии vCenter: 7.0 и выше
  \item Поддерживаемые версии ESXi: 7.0 и выше
  \item Режим выполнения: Интерактивный (через консоль)
  \item Время настройки vSwitch: 5–30 секунд
  \item Используемый протокол подключения: HTTPS с TLS 1.2+
  \item Поддержка MTU: Автоматическая установка MTU = 9000
  \item Копируемые параметры: Названия порт-групп, VLAN ID, политики безопасности
\end{itemize}

\newpage
\section{Комплектность}

В состав программного изделия «Программа конфигурации хостов виртуализации» (ЛАБ 1.ПК-ПУПС) входят следующие компоненты:

\begin{center}
\textbf{Форма 1. Комплектность}
\end{center}

\begin{tabular}{|p{4cm}|p{5.5cm}|p{2cm}|p{3cm}|p{2.5cm}|}
\hline
\textbf{Обозначение} & \textbf{Наименование} & \textbf{Количество} & \textbf{Порядковый учётный номер} & \textbf{Примечание} \\
\hline
ЛАБ 1.ПК-ПУПС & Скрипт PowerShell конфигурации ESXi-хостов & 1 & 001 & Основной исполняемый файл \\
\hline
ЛАБ 1.ТЗ & Техническое задание & 1 & 002 & Документация \\
\hline
ЛАБ 1.ПЗ & Пояснительная записка & 1 & 003 & Документация \\
\hline
ЛАБ 1.РСП & Руководство системного программиста & 1 & 004 & Документация \\
\hline
ЛАБ 1.РО & Руководство оператора & 1 & 005 & Документация \\
\hline
ЛАБ 1.ИН & Инструкция по установке и наладке & 1 & 006 & Документация \\
\hline
ЛАБ 1.ПМИ & Программа и методика испытаний & 1 & 007 & Документация \\
\hline
ЛАБ 1.ФО & Формуляр & 1 & 008 & Настоящий документ \\
\hline
ЛАБ 1.ВЭД & Ведомость эксплуатационных документов & 1 & 009 & Сводная ведомость \\
\hline
\end{tabular}


\newpage
