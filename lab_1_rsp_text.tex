%\documentclass[12pt,a4paper]{article}
%\usepackage{tech_doc_rus}
%\makenomenclature %Закомментировать, если не нужен

%\begin{document}

%\tableofcontents

%\newpage
%\addcontentsline{toc}{section}{\tocsecindent{Обозначения и сокращения}}
%\renewcommand{\nomname}{Обозначения и сокращения}
%\printnomenclature
%\newpage

%\section*{Аннотация}


\makenomenclature

\nomenclature{ПО}{Программное обеспечение --- Совокупность программ, используемых для выполнения определённых задач на вычислительной системе}
\nomenclature{ESXi}{VMware ESXi --- Гипервизор, обеспечивающий виртуализацию на уровне ядра операционной системы}
\nomenclature{vCenter}{vCenter Server --- Центральный сервер управления инфраструктурой виртуализации от VMware}
\nomenclature{vSwitch}{Virtual Switch --- Виртуальный коммутатор, обеспечивающий сетевое взаимодействие между виртуальными машинами и физической сетью}
\nomenclature{VLAN ID}{Virtual Local Area Network Identifier --- Идентификатор виртуальной локальной сети, используемый для разделения трафика внутри одной физической сети}
\nomenclature{MTU}{Maximum Transmission Unit --- Максимальный размер передаваемого сетевого пакета, измеряется в байтах}
\nomenclature{PowerCLI}{VMware PowerCLI --- Набор PowerShell-модулей для управления инфраструктурой VMware через скрипты}
\nomenclature{PowerShell 7}{PowerShell Core 7 --- Кроссплатформенная версия PowerShell, поддерживающая современные функции и .NET Core}
\nomenclature{Port Group}{Порт-группа --- Логическая группа портов на виртуальном коммутаторе, имеющая общие параметры VLAN и безопасности}
\nomenclature{Maintenance Mode}{Режим обслуживания --- Состояние хоста ESXi, при котором он временно исключается из активной эксплуатации для проведения работ}
\nomenclature{TLS}{Transport Layer Security --- Протокол шифрования, обеспечивающий безопасную передачу данных по сети}
\nomenclature{HTTPS}{HyperText Transfer Protocol Secure --- Безопасный протокол передачи данных поверх TLS}
\nomenclature{CI/CD}{Continuous Integration / Continuous Delivery --- Методологии автоматизации разработки и доставки программного обеспечения}
\nomenclature{Git}{Git --- Система контроля версий, используемая для отслеживания изменений в коде программы}

\newpage\annotation

Настоящее \textit{Руководство системного программиста} предназначено для ознакомления с назначением, структурой, особенностями настройки и проверкой работоспособности программного средства, а также для оказания помощи системному программисту при установке, сопровождении и диагностике программы. Документ составлен в соответствии с требованиями ГОСТ 19.503–79 и включает в себя полное описание всех аспектов, необходимых для эксплуатации программы в различных условиях.

В разделе «Общие сведения о программе» представлены сведения о назначении и реализуемых функциях программы, а также указаны требования к техническим и программным средствам, обеспечивающим корректное выполнение программы. В разделе «Структура программы» подробно раскрыта логическая структура программного обеспечения, описаны составные части, взаимосвязи между ними и зависимости от внешних программ.

Раздел «Настройка программы» содержит последовательное описание действий, необходимых для адаптации программы к конкретной вычислительной среде, включая настройку под конкретный состав оборудования и выбор конфигурационных параметров. При необходимости приводятся примеры настройки.

Раздел «Проверка программы» описывает методы и средства, с помощью которых осуществляется контроль работоспособности программы. Приведены типовые контрольные примеры, методы их использования и возможные результаты проверки.

В разделе «Дополнительные возможности» изложено описание расширенных функций программы, доступных при определённых условиях, а также порядок их активации и применения.

Раздел «Сообщения системному программисту» содержит перечень диагностических и информационных сообщений, которые могут быть выведены программой на различных этапах её использования. Даётся расшифровка содержания сообщений и указания по действиям, которые должен предпринять системный программист при их появлении.

В приложении представлены вспомогательные материалы: примеры выполнения операций, иллюстрации структуры, таблицы конфигурационных параметров и иные данные, способствующие более глубокому пониманию принципов работы программы и её сопровождению.

\newpage
\section{Общие сведения о программе}

Программа предназначена для автоматизации переноса сетевой конфигурации между хостами виртуализации VMware ESXi, находящимися под управлением VMware vCenter. Основной задачей является копирование параметров виртуальных коммутаторов (vSwitch) и связанных с ними порт-групп с одного хоста на другой с целью ускорения процессов настройки и унификации сетевой инфраструктуры в пределах виртуализированной среды.

Программа реализована в виде сценария PowerShell с использованием модуля VMware.PowerCLI, что позволяет интегрировать её в существующие процессы администрирования и использовать на платформах, поддерживающих PowerShell 7.x и выше.

Выполнение программы обеспечивается следующими техническими и программными средствами:
\begin{itemize}
  \item Операционная система: Windows 10/11 или Windows Server (с поддержкой PowerShell 7.x)
  \item Среда выполнения: PowerShell 7.x
  \item Дополнительные модули: VMware.PowerCLI версии не ниже 12.0
  \item Подключение к серверу управления VMware vCenter с необходимыми правами доступа
\end{itemize}

Программа предназначена для использования системными администраторами, инженерами по виртуализации и специалистами по информационной безопасности, осуществляющими миграцию, резервирование или восстановление конфигурации хостов ESXi в пределах центра обработки данных (ЦОД) или испытательного стенда.

\newpage
\section{Структура программы}

Программа реализована в виде модуля PowerShell-скрипта, структурированного на логически обособленные функциональные блоки. Каждый блок отвечает за выполнение конкретной задачи и взаимодействует с другими через передачу параметров и возврат значений. Это обеспечивает модульность, удобство сопровождения и возможность масштабирования функционала.

Основные составные части программы:
\begin{enumerate}
  \item \textbf{Модуль инициализации окружения} — отвечает за проверку наличия необходимых зависимостей (в первую очередь, модуля VMware.PowerCLI), инициализацию переменных и подключение к vCenter.
  \item \textbf{Блок взаимодействия с пользователем} — реализует ввод параметров (имена исходного и целевого хостов, выбор vSwitch'ей, VLAN и прочее), обеспечивает валидацию введённых данных.
  \item \textbf{Модуль получения конфигурации исходного хоста} — собирает данные о текущей конфигурации vSwitch'ей и связанных с ними порт-групп на исходном хосте.
  \item \textbf{Модуль копирования и настройки конфигурации} — создает или модифицирует vSwitch'и и порт-группы на целевом хосте в соответствии с полученной ранее конфигурацией.
  \item \textbf{Модуль проверки корректности выполнения} — включает в себя функции логирования, отслеживания ошибок и формирования итогового отчета.
\end{enumerate}

Связи между составными частями носят линейно-итеративный характер: выполнение одного модуля инициирует вызов следующего, при этом возможна циклическая обработка ошибок и повторный ввод данных пользователем.

Связи с другими программами:
\begin{itemize}
  \item \textbf{VMware.PowerCLI} — используется как основная библиотека взаимодействия с API vCenter/ESXi.
  \item \textbf{Средства логирования (опционально)} — программа может быть дополнена внешними средствами сбора логов (например, Logstash, syslog), при расширении функционала.
\end{itemize}

Такая структура позволяет гибко адаптировать программу под специфические задачи эксплуатации и обеспечивать её устойчивость и расширяемость в рамках инфраструктурных требований.

\newpage
\section{Настройка программы}

Для корректной работы программы необходимо произвести предварительную настройку, включающую подготовку технического и программного окружения, а также задание параметров выполнения. Настройка направлена на адаптацию программы к конкретной инфраструктуре, включая состав технических средств и используемую версию ПО.

\subsection{Подготовка технической среды}
Перед запуском программы необходимо обеспечить доступ к следующей инфраструктуре:
\begin{itemize}
  \item Сервер vCenter с доступом по сети (рекомендуется использовать FQDN);
  \item По крайней мере два ESXi-хоста, подключенных к vCenter и находящихся в рабочем состоянии;
  \item Рабочая станция с установленной ОС Windows 10/11 или Windows Server, поддерживающей запуск PowerShell 7.x.
\end{itemize}

\subsection{Программные компоненты}
Для выполнения программы требуется:
\begin{itemize}
  \item PowerShell 7.0 или выше;
  \item Установленный и корректно загруженный модуль \texttt{VMware.PowerCLI} версии не ниже 12.0;
  \item Опционально — средства логирования и мониторинга (например, вывод в файл, интеграция с SIEM).
\end{itemize}

Пример команды установки PowerCLI:
\begin{verbatim}
Install-Module -Name VMware.PowerCLI -Scope CurrentUser
\end{verbatim}

\subsection{Выбор функции и параметров}
При запуске пользователь последовательно выбирает параметры:
\begin{enumerate}
  \item Адрес vCenter и учетные данные для подключения.
  \item Исходный и целевой ESXi-хосты.
  \item vSwitch и порт-группы, подлежащие копированию.
  \item Дополнительные параметры (например, MTU, VLAN ID).
\end{enumerate}

\subsection{Пример настройки}
\textbf{Сценарий:} копирование сетевой конфигурации с хоста \texttt{esxi-01} на \texttt{esxi-02}:
\begin{enumerate}
  \item Запустить PowerShell с правами администратора.
  \item Подключиться к vCenter: \texttt{Connect-VIServer -Server vcenter.lab.local}.
  \item Ввести имя исходного хоста: \texttt{esxi-01.lab.local}.
  \item Ввести имя целевого хоста: \texttt{esxi-02.lab.local}.
  \item Указать нужный vSwitch: \texttt{vSwitch1}.
  \item Подтвердить копирование порт-групп.
\end{enumerate}

После настройки программа готова к выполнению основной функциональности. При необходимости конфигурационные параметры могут быть заданы через параметры запуска или внешние конфигурационные файлы (в будущем).

\newpage
\section{Проверка программы}

Проверка программы проводится с целью подтверждения её работоспособности и соответствия функциональным требованиям, указанным в техническом задании. В процессе проверки выполняются контрольные запуски программы с заранее заданными входными данными, оценивается корректность выполнения операций и сопоставляются фактические результаты с ожидаемыми.

\subsection{Методы проверки}

Для проверки программы применяются следующие методы:
\begin{itemize}
  \item функциональное тестирование на подготовленных стендах;
  \item прогон с контрольными данными, моделирующими типовые сценарии использования;
  \item проверка устойчивости к ошибкам и некорректному вводу;
  \item ручная верификация выводимых сообщений и итогов выполнения операций.
\end{itemize}

\subsection{Контрольные примеры}

Пример 1: Копирование vSwitch и порт-групп с одного хоста на другой.

\begin{itemize}
  \item \textbf{Исходный хост:} \texttt{esxi-01.lab.local}
  \item \textbf{Целевой хост:} \texttt{esxi-02.lab.local}
  \item \textbf{vSwitch:} \texttt{vSwitch1}
  \item \textbf{Ожидаемый результат:} На целевом хосте создаются те же порт-группы с аналогичными параметрами (имя, VLAN ID, тип подключения), без изменения MTU, если он уже совпадает.
\end{itemize}

Пример 2: Проверка обработки некорректного ввода.

\begin{itemize}
  \item \textbf{Условие:} указание несуществующего имени хоста.
  \item \textbf{Ожидаемый результат:} программа выводит сообщение об ошибке и запрашивает повторный ввод, без аварийного завершения.
\end{itemize}

\subsection{Результаты проверки}

В ходе испытаний оцениваются следующие критерии:
\begin{enumerate}
  \item Все заявленные функции выполняются в соответствии с описанием в техническом задании.
  \item Программа не завершает выполнение аварийно при штатных и ошибочных сценариях.
  \item Все сообщения понятны, отражают суть выполняемых действий или возникших проблем.
  \item Логика взаимодействия с пользователем последовательна и предсказуема.
\end{enumerate}

При успешном выполнении всех контрольных примеров программа признается работоспособной и готовой к эксплуатации в тестовой или промышленной среде.

\newpage
\section{Дополнительные возможности}

Программа предусматривает ряд дополнительных функциональных возможностей, расширяющих её основное назначение и повышающих удобство эксплуатации системным программистом. Эти возможности не являются обязательными для базового сценария использования, но могут быть активированы при необходимости.

\subsection{Поддержка различных версий PowerCLI и PowerShell}

Программа может быть запущена в средах с различными версиями PowerShell (включая PowerShell 7.x) и VMware PowerCLI (от версии 12.0 и выше). В зависимости от версии PowerCLI автоматически адаптируются вызовы командлетов и обрабатываются различия в API.

\subsection{Интерактивный режим работы}

В интерактивном режиме программа:
\begin{itemize}
  \item запрашивает у пользователя параметры подключения и имена хостов;
  \item предлагает выбор из доступных vSwitch’ей;
  \item подтверждает критические действия перед их выполнением (например, изменение MTU).
\end{itemize}

Этот режим может быть активирован по умолчанию или посредством специального ключа командной строки (например, `-Interactive`).

\subsection{Возможность расширения логирования}

В текущей реализации программа выводит ключевую информацию в консоль, однако архитектура предусматривает возможность подключения внешнего логгера для записи событий в файл. Это особенно полезно для аудита, отладки и сопровождения.

\subsection{Гибкая настройка параметров}

При необходимости параметры запуска (например, имя vCenter, учетные данные, имена хостов и т.д.) могут быть переданы через конфигурационный файл или аргументы командной строки. Это позволяет использовать программу в автоматизированных скриптах и сценариях CI/CD.

\subsection{Модульность и расширяемость}

Код программы организован по модульному принципу, что позволяет легко добавлять новые функции, такие как:
\begin{itemize}
  \item клонирование стандартных порт-групп в распределённые (vDS),
  \item настройка политик безопасности в рамках порт-групп,
  \item проверка соответствия конфигурации заданным шаблонам.
\end{itemize}

Таким образом, дополнительные возможности делают программу гибкой и адаптируемой к различным условиям эксплуатации, в том числе в больших инфраструктурах виртуализации.

\newpage
\section{Сообщения системному программисту}

\section{Сообщения системному программисту}

В процессе работы программа генерирует ряд информационных, предупреждающих и ошибок сообщений, предназначенных для информирования системного программиста о текущем состоянии, обнаруженных проблемах и необходимых действиях. Ниже приведены типовые сообщения, их значение и рекомендации по реагированию.

\subsection{Информационные сообщения}

\begin{itemize}
  \item [INFO] Подключение к vCenter (указано имя сервера) успешно установлено.\\
  Означает, что соединение с сервером vCenter прошло успешно. Действия не требуются.

  \item [INFO] Режим обслуживания включён на хосте (указано имя хоста).\\
  Хост переведён в maintenance mode, можно безопасно вносить изменения. Убедитесь, что виртуальные машины предварительно мигрированы.

  \item [INFO] Копирование порт-групп завершено успешно.\\
  Все сетевые параметры были успешно перенесены. Дополнительных действий не требуется.
\end{itemize}

\subsection{Предупреждающие сообщения}

\begin{itemize}
  \item [WARN] Порт-группа с заданным именем уже существует на целевом хосте.\\
  Программа обнаружила конфликт имён. Рекомендуется проверить существующую конфигурацию и принять решение о перезаписи или переименовании.

  \item [WARN] Значение MTU отличается от рекомендованного (1500).\\
  Возможна проблема с производительностью. При необходимости измените MTU вручную.

  \item [WARN] Модуль VMware PowerCLI не найден. Попытка установить.\\
  Если установка завершится неудачно, необходимо вручную установить модуль и перезапустить выполнение.
\end{itemize}

\subsection{Сообщения об ошибках}

\begin{itemize}
  \item [ERROR] Не удалось подключиться к vCenter: неверные учётные данные.\\
  Проверьте логин и пароль. При необходимости повторите попытку с корректными данными.

  \item [ERROR] Хост не найден.\\
  Указанный хост отсутствует в инфраструктуре. Убедитесь в корректности написания имени и наличии доступа к нужному кластеру.

  \item [ERROR] Ошибка при создании vSwitch: недостаточно прав.\\
  Учетная запись, под которой выполняется программа, не обладает необходимыми правами. Проверьте роль пользователя в vCenter.

  \item [FATAL] Критическая ошибка при инициализации. Выполнение остановлено.\\
  Возникла непредвиденная ошибка, требующая анализа логов. Рекомендуется обратиться к разработчику или системному администратору.
\end{itemize}

\subsection{Рекомендации по реагированию}

При получении предупреждающих или ошибочных сообщений системному программисту следует:

\begin{enumerate}
  \item Проанализировать сообщение и его контекст.
  \item Проверить корректность входных параметров.
  \item Ознакомиться с логами программы для получения дополнительной информации.
  \item При системных сбоях — уведомить администратора или инициировать отладку.
\end{enumerate}

Такая организация обработки сообщений обеспечивает устойчивость и надёжность эксплуатации программного обеспечения.
