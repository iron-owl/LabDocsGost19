\newcommand{\producttype}{Программное обеспечение}
\newcommand{\productcipher}{ЛАБ 1} % Шифр изделия
\newcommand{\productfullname}{Программа конфигурации хостов виртуализации}

\iulversion{1}
\iulnumoflastchange{0}
\iulnote{~}

\firstapplication{\productcipher} % Первое применение
\referencenumber{\productcipher} % Справ №

\redaction{01} % Номер редакции
\documentnumber{03} % Номер документа данного вида
%\partnumber{1} % Номер части документа - чтобы задать - надо фиксить класс espd

\iuldeveloper{Иванов И. И.} % Разраб.
\iulcheck{Аникаев К. П.} % Проверил
\iulnormocontrol{Аникаев К. П.} % Н.контр
\iulapproved{Аникаев К. П.} % Утв.
\organizationcode{01132212} % Код организации / ОКПО
\registrationcode{62.01.12} % Регистрационный код / окпд 2
% СОГЛАСОВАНО
\customerrank{Старший преподаватель кафедры ОРТЗИ}
\customername{Аникаев К. П.}

% УТВЕРЖДАЮ
\chiefconstructorrank{Старший преподаватель кафедры ОРТЗИ}
\chiefconstructorname{Аникаев К. П.}

\fromcustomerrank{Старший преподаватель кафедры ОРТЗИ}
\fromcustomername{Аникаев К. П.}

\headofdepartmentrank{Старший преподаватель кафедры ОРТЗИ}
\headofdepartmentname{Аникаев К. П.}

\deputyofchiefconstructorrank{Старший преподаватель кафедры ОРТЗИ}
\deputyofchiefconstructorname{Бармотин А. Д.}
%
\developerrank{Старший преподаватель кафедры ОРТЗИ}
\developername{Аникаев К. П.}

%\headoflaboratoryrank{Начальник лаборатории 777}
%\headoflaboratoryname{А.Б.~Лабораториев}

% Исполнитель(и)
\authorname{Иванов И. И.}
%\authori{Новиков~А.~Н.}
%\authorii{Михайлов~Е.~Н.}

\year{2025}

