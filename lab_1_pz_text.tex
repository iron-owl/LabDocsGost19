%\documentclass[12pt,a4paper]{article}
%\usepackage{tech_doc_rus}
%\makenomenclature %Закомментировать, если не нужен

%\begin{document}

%\tableofcontents

%\newpage
%\addcontentsline{toc}{section}{\tocsecindent{Обозначения и сокращения}}
%\renewcommand{\nomname}{Обозначения и сокращения}
%\printnomenclature
%\newpage

%\section*{Аннотация}


\makenomenclature

\nomenclature{ПО}{Программное обеспечение --- Совокупность программ, используемых для выполнения определённых задач на вычислительной системе}
\nomenclature{ESXi}{VMware ESXi --- Гипервизор, обеспечивающий виртуализацию на уровне ядра операционной системы}
\nomenclature{vCenter}{vCenter Server --- Центральный сервер управления инфраструктурой виртуализации от VMware}
\nomenclature{vSwitch}{Virtual Switch --- Виртуальный коммутатор, обеспечивающий сетевое взаимодействие между виртуальными машинами и физической сетью}
\nomenclature{VLAN ID}{Virtual Local Area Network Identifier --- Идентификатор виртуальной локальной сети, используемый для разделения трафика внутри одной физической сети}
\nomenclature{MTU}{Maximum Transmission Unit --- Максимальный размер передаваемого сетевого пакета, измеряется в байтах}
\nomenclature{PowerCLI}{VMware PowerCLI --- Набор PowerShell-модулей для управления инфраструктурой VMware через скрипты}
\nomenclature{PowerShell 7}{PowerShell Core 7 --- Кроссплатформенная версия PowerShell, поддерживающая современные функции и .NET Core}
\nomenclature{Port Group}{Порт-группа --- Логическая группа портов на виртуальном коммутаторе, имеющая общие параметры VLAN и безопасности}
\nomenclature{Maintenance Mode}{Режим обслуживания --- Состояние хоста ESXi, при котором он временно исключается из активной эксплуатации для проведения работ}
\nomenclature{TLS}{Transport Layer Security --- Протокол шифрования, обеспечивающий безопасную передачу данных по сети}
\nomenclature{HTTPS}{HyperText Transfer Protocol Secure --- Безопасный протокол передачи данных поверх TLS}
\nomenclature{CI/CD}{Continuous Integration / Continuous Delivery --- Методологии автоматизации разработки и доставки программного обеспечения}
\nomenclature{Git}{Git --- Система контроля версий, используемая для отслеживания изменений в коде программы}

\newpage\annotation

Настоящий документ представляет собой пояснительную записку к программе \textbf{«Программа конфигурации хостов виртуализации»} (шифр \textit{ЛАБ 1}), предназначенной для автоматизации сетевой настройки хостов VMware ESXi при их добавлении или восстановлении в инфраструктуре внутреннего облачного окружения.

Программа реализована в виде PowerShell-скрипта и использует модуль \texttt{VMware PowerCLI} для взаимодействия с серверами vCenter. Основными функциями программы являются:
\begin{itemize}
\item проверка наличия и параметров виртуальных коммутаторов (\textit{vSwitch});
\item установка корректного значения MTU (\textbf{9000});
\item копирование порт-групп с исходного хоста на целевой, включая VLAN ID и политики безопасности.
\end{itemize}

Документ разработан в соответствии с требованиями ГОСТ~19.404--79 и содержит описание назначения, области применения, технических характеристик программы, а также информацию об ожидаемой эффективности её использования.

Пояснительная записка предназначена для ознакомления следующих категорий специалистов:
\begin{enumerate}
    \item системные администраторы;
    \item DevOps-инженеры;
    \item сотрудники технической поддержки;
    \item заказчики программного обеспечения.
\end{enumerate}

Цель данного документа --- предоставить полное представление о программе, её архитектуре, особенностях реализации и условиях применения.

\newpage
\section{Введение}

Настоящая пояснительная записка разработана на программу «Программа конфигурации хостов виртуализации» (шифр проекта: ЛАБ 1), предназначенную для автоматизации процессов сетевой настройки хостов VMware ESXi при их добавлении или восстановлении в инфраструктуре внутреннего облачного окружения.

Разработка программы выполнена в соответствии с техническим заданием, подготовленным специалистом Kirill Anikaev 10 февраля 2024 года. Целью создания программы является обеспечение унифицированной и надёжной настройки виртуальных коммутаторов (vSwitch) и порт-групп на новых или восстанавливаемых ESXi-хостах в соответствии с внутренним административным руководством компании.

Программа реализована в виде PowerShell-скрипта и использует модуль VMware PowerCLI для взаимодействия с серверами vCenter. Она позволяет системным администраторам значительно ускорить процессы настройки сетевой подсистемы, избежать ошибок, связанных с ручным вводом, а также обеспечивает единые параметры сети (включая MTU и VLAN ID) для всех хостов.

Документ предназначен для использования системными администраторами, DevOps-инженерами, сотрудниками технической поддержки и заказчиками программного обеспечения.
\newpage
\section{Назначение и область применения}

Программа \textbf{«Программа конфигурации хостов виртуализации»} (шифр \textit{ЛАБ 1}) предназначена для автоматизации процесса настройки сетевой инфраструктуры при добавлении новых или восстановлении существующих хостов VMware ESXi в инфраструктуре внутреннего облачного окружения.

\subsection{Цель использования программы}

Основной целью создания программы является обеспечение унифицированной и надёжной конфигурации виртуальных коммутаторов (\textit{vSwitch}) и порт-групп на хостах ESXi. Программа позволяет избежать ошибок, связанных с ручным вводом данных, а также обеспечивает соблюдение стандартов сети (включая параметры MTU и VLAN ID) на всех хостах.

\subsection{Решаемые задачи}
Программа решает следующие задачи:
\begin{itemize}
\item проверка наличия виртуального коммутатора на целевом хосте;
\item проверка и установка значения MTU равного \textbf{9000};
\item создание отсутствующих виртуальных коммутаторов;
\item копирование порт-групп с исходного хоста на целевой;
\item сохранение параметров VLAN ID и политик безопасности при копировании.
\end{itemize}

\subsection{Область применения}
Программа применяется в следующих областях:
\begin{enumerate}
    \item Настройка новых ESXi-хостов при расширении инфраструктуры виртуализации;
    \item Восстановление хостов после аппаратных сбоев или переустановки гипервизора;
    \item Стандартизация сетевых параметров в рамках внутреннего облака компании;
    \item Автоматизация рутинных операций системными администраторами и DevOps-инженерами.
\end{enumerate}

\subsection{Условия эксплуатации}
Для корректной работы программы необходимо следующее окружение:
\begin{itemize}
\item Операционная система: Windows (с поддержкой PowerShell 7);
\item Доступ к серверам vCenter через протокол HTTPS.
\end{itemize}

\subsection{Категории пользователей}
Программа предназначена для использования следующими категориями специалистов:
\begin{itemize}
\item \textbf{Системные администраторы} — для настройки и восстановления хостов;
\item \textbf{DevOps-инженеры} — для интеграции скрипта в процессы автоматизации;
\item \textbf{Специалисты технической поддержки} — для устранения проблем с сетевой конфигурацией;
\item \textbf{Заказчики ПО} — для ознакомления с функционалом и возможностями программы.
\end{itemize}
\newpage

\section{Технические характеристики}
\subsection{Постановка задачи}

Целью разработки программы <<Программа конфигурации хостов виртуализации>> (шифр ЛАБ 1) является автоматизация процесса настройки сетевой подсистемы хостов VMware ESXi при их добавлении или восстановлении в инфраструктуре внутреннего облачного окружения.

\subsubsection{Исходные данные}
Программа предназначена для работы в среде PowerShell с использованием модуля VMware PowerCLI, обеспечивая взаимодействие с серверами vCenter. Основными объектами взаимодействия являются:

\begin{itemize}
    \item серверы vCenter;
    \item целевые и исходные хосты ESXi;
    \item виртуальные коммутаторы (vSwitch);
    \item порт-группы с VLAN ID.
\end{itemize}

\subsubsection{Формулировка задачи}
Задачей программы является обеспечение корректной и стандартизированной настройки сетевых параметров на новых или восстанавливаемых ESXi-хостах, включая следующие действия:

\begin{itemize}
    \item Проверка наличия виртуального коммутатора на целевом хосте.
    \item Проверка и при необходимости изменение значения MTU на 9000.
    \item Создание отсутствующего виртуального коммутатора.
    \item Копирование порт-групп с исходного хоста на целевой.
    \item Сохранение VLAN ID и политики безопасности при копировании.
\end{itemize}

\subsubsection{Требования технического задания}

В рамках реализации были учтены следующие требования:

\begin{itemize}
    \item возможность выбора сервера vCenter из предложенного списка;
    \item проверка режима обслуживания целевого хоста;
    \item интерактивный диалог с пользователем для выбора vSwitch'а и действий над ним;
    \item возможность автоматического создания vSwitch'а и установки MTU = 9000;
    \item копирование порт-групп с сохранением всех сетевых параметров.
\end{itemize}

\subsubsection{Ожидаемый результат}
В результате выполнения программы достигается унифицированная сетевая конфигурация хостов ESXi, соответствующая стандартам внутренней инфраструктуры компании. Это позволяет:

\begin{itemize}
    \item значительно сократить время настройки хостов;
    \item минимизировать вероятность ошибок при ручной настройке;
    \item обеспечить единые сетевые параметры для всех хостов облака;
    \item повысить качество эксплуатации и обслуживания виртуальной инфраструктуры.
\end{itemize}

\subsection{Математические методы и модели}

\subsubsection{Общие положения}
Программа <<Программа конфигурации хостов виртуализации>> (шифр ЛАБ 1) не предполагает использование сложных математических моделей, так как её функционал направлен на автоматизацию сетевых операций в среде VMware ESXi через PowerShell и PowerCLI. Однако для реализации задачи используются логические алгоритмы и структуры данных, обеспечивающие корректное взаимодействие с виртуальной инфраструктурой.

Все действия программы строятся на основе последовательной проверки состояния объектов виртуальной среды и выполнении действий по изменению их параметров согласно заданным условиям.

\subsubsection{Основные принципы работы}
Программа использует следующие подходы при обработке информации:

\begin{itemize}
    \item Последовательная проверка наличия vCenter, подключение к выбранному серверу.
    \item Проверка режима обслуживания целевого хоста перед началом конфигурационных работ.
    \item Интерактивный диалог с пользователем, позволяющий выбрать нужный vSwitch.
    \item Анализ текущего состояния виртуального коммутатора (наличие, значение MTU).
    \item Выполнение действий в зависимости от состояния: создание, изменение или пропуск.
\end{itemize}

\subsubsection{Используемые логические конструкции}
Программа основана на применении условных выражений и циклов:

\begin{itemize}
    \item Условные операторы (if/else) — для проверки наличия модулей, существования vSwitch, режима обслуживания хоста и других параметров.
    \item Циклы (do-while) — для повторного выбора vSwitch'а до тех пор, пока пользователь не примет решение выйти из скрипта.
    \item Функции — для упрощения повторяющихся действий, таких как проверка наличия и MTU vSwitch'а.
\end{itemize}

\subsubsection{Взаимодействие с API vSphere}
Скрипт использует интерфейсы VMware vSphere SDK через PowerCLI, что позволяет:

\begin{itemize}
    \item Получать информацию о состоянии хостов и сетевой инфраструктуры.
    \item Управлять виртуальными коммутаторами и порт-группами.
    \item Применять политики безопасности и VLAN ID.
\end{itemize}

При работе с API применяются стандартные методы получения и изменения объектов:

\begin{itemize}
    \item \texttt{Get-VMHost} — получение информации о хостах.
    \item \texttt{Get-VirtualSwitch}, \texttt{New-VirtualSwitch}, \texttt{Set-VirtualSwitch} — работа с виртуальными коммутаторами.
    \item \texttt{Get-VirtualPortGroup}, \texttt{New-VirtualPortGroup} — управление порт-группами.
\end{itemize}

\subsubsection{Допущения и ограничения}
При разработке были приняты следующие допущения:

\begin{itemize}
    \item Предполагается, что целевой хост находится в режиме обслуживания.
    \item Значение MTU всегда должно быть равно 9000 для соответствия внутренним стандартам сети.
    \item Сетевые параметры исходного хоста считаются эталонными и подлежат копированию без изменений.
\end{itemize}

К основным ограничениям относятся:

\begin{itemize}
    \item Программа предназначена только для работы в среде Windows.
    \item Необходимо наличие установленного модуля VMware PowerCLI.
    \item Поддерживается только протокол HTTPS для подключения к vCenter.
    \item Работа возможна только с хостами, находящимися в режиме обслуживания.
\end{itemize}


\subsection{Алгоритм работы программы}

\subsubsection{Общее описание}
Алгоритм работы программы <<Программа конфигурации хостов виртуализации>> (шифр ЛАБ 1) представляет собой последовательность действий, направленных на автоматизацию сетевой настройки новых или восстанавливаемых хостов VMware ESXi.

Программа реализована в виде PowerShell-скрипта и использует модуль VMware PowerCLI для взаимодействия с серверами vCenter. Основная цель алгоритма --- обеспечить унифицированную настройку виртуальных коммутаторов (vSwitch) и порт-групп с сохранением VLAN ID и политик безопасности.

\subsubsection{Структура алгоритма}

Структура алгоритма представлена в виде блок-схемы на рис.~(\ref{lab_1_block_scheme.drawio}).

\illustration[][Структура алгоритма][0.5]{lab_1_block_scheme.drawio.pdf}[lab_1_block_scheme.drawio]

\subsubsection{Особенности реализации}
\begin{itemize}
    \item \textbf{Интерактивность}: большинство действий выполняется после подтверждения пользователя.
    \item \textbf{Многократное использование}: цикл позволяет обработать несколько vSwitch'ей за один запуск программы.
    \item \textbf{Устойчивость к ошибкам}: предусмотрена обработка ошибок при подключении к серверам и некорректном вводе данных.
\end{itemize}

\subsection{Взаимодействие с другими программами}

\subsubsection{Общие сведения}
Программа <<Программа конфигурации хостов виртуализации>> (шифр ЛАБ 1) представляет собой PowerShell-скрипт, предназначенный для автоматизации сетевых настроек виртуальных хостов VMware ESXi. Для выполнения своих функций программа взаимодействует с рядом внешних программных компонентов и сервисов.

Основное взаимодействие осуществляется через \textbf{VMware PowerCLI}, который обеспечивает доступ к API vSphere и позволяет управлять объектами инфраструктуры виртуализации.

\subsubsection{Основные компоненты взаимодействия}
\begin{enumerate}
    \item \textbf{VMware vCenter Server}
    \begin{itemize}
        \item Центральный управляющий компонент инфраструктуры VMware.
        \item Основные действия:
        \begin{itemize}
            \item Подключение к серверу vCenter.
            \item Получение списка хостов.
            \item Чтение и изменение параметров сетевой конфигурации.
        \end{itemize}
        \item Используемые команды:
        \begin{itemize}
            \item \texttt{Connect-VIServer} — установление соединения.
            \item \texttt{Get-VMHost} — информация о хостах.
            \item \texttt{Get-VirtualSwitch}, \texttt{Get-VirtualPortGroup} — работа с сетью.
        \end{itemize}
    \end{itemize}
    
    \item \textbf{VMware ESXi Host}
    \begin{itemize}
        \item Целевой объект настройки
        \item Основные действия:
        \begin{itemize}
            \item Проверка режима обслуживания.
            \item Модификация виртуальных коммутаторов.
            \item Копирование порт-групп с сохранением параметров.
        \end{itemize}
    \end{itemize}
    
    \item \textbf{Модуль VMware PowerCLI}
    \begin{itemize}
        \item Набор PowerShell-модулей для управления VMware
        \item Установка модуля:
\begin{lstlisting}[language=PowerShell, basicstyle=\small\ttfamily]
Install-Module -Name VMware.PowerCLI `
               -Confirm:$false `
               -AllowClobber `
               -Force
\end{lstlisting}
    \end{itemize}
\end{enumerate}

\subsubsection{Дополнительные компоненты}
\begin{itemize}
    \item \textbf{Системы контроля версий} (Git) — для управления исходным кодом.
    \item \textbf{Средства мониторинга} (Zabbix, Grafana) — для отслеживания выполнения.
    \item \textbf{Другие скрипты} — интеграция в процессы автоматизации.
\end{itemize}

\subsubsection{Ограничения взаимодействия}
\begin{itemize}
    \item Требуется Windows с PowerShell 7.
    \item Необходимы права администратора vCenter.
    \item Обязателен режим обслуживания хоста.
    \item Поддерживается только HTTPS-подключение.
\end{itemize}

\subsection{Организация входных и выходных данных}

\subsubsection{Общие положения}
Программа <<Программа конфигурации хостов виртуализации>> (шифр ЛАБ 1) представляет собой PowerShell-скрипт, ориентированный на автоматизацию сетевых настроек ESXi-хостов. Для выполнения своих функций программа использует как \textbf{входные данные}, вводимые пользователем или поступающие из внешних источников, так и формирует \textbf{выходные данные}, предназначенные для отслеживания хода выполнения и результатов работы.

Все операции с данными происходят в интерактивном режиме через командную строку PowerShell 7.

\subsubsection{Входные данные}
Входные данные обеспечивают программе необходимую информацию для выполнения задач. Они делятся на следующие категории:

\begin{enumerate}
    \item \textbf{Пользовательский ввод.}
    \begin{itemize}
        \item Выбор vCenter — пользователь выбирает номер сервера vCenter из предложенного списка.
        \item Имя целевого хоста — указывается имя нового или восстанавливаемого ESXi-хоста (без домена), который находится в режиме обслуживания.
        \item Имя исходного хоста — указывается имя существующего ESXi-хоста, с которого будут копироваться порт-группы.
        \item Выбор vSwitch — пользователь указывает имя виртуального коммутатора (например, vSwitch0), параметры которого нужно проверить или изменить.
        \item Подтверждение действий — на каждом этапе (создание vSwitch, изменение MTU, копирование порт-групп) пользователь подтверждает действие символом Y/y или отменяет его N/n.
    \end{itemize}
    
    \item \textbf{Конфигурационные данные.}
    \begin{itemize}
        \item Список доступных серверов vCenter.
        \item Домен, добавляемый к имени хоста.
        \item Значение MTU (всегда устанавливается равным 9000).
        \item Список политик безопасности порт-групп.
    \end{itemize}
    
    \item \textbf{Данные из vCenter/ESXi.}
    \begin{itemize}
        \item Наличие и состояние vSwitch'ей на целевом хосте.
        \item Значение MTU у каждого vSwitch.
        \item Список порт-групп на исходном хосте.
        \item VLAN ID и политики безопасности порт-групп.
    \end{itemize}
\end{enumerate}

\subsubsection{Выходные данные}
\begin{enumerate}
    \item \textbf{Информационные сообщения.}
    \begin{itemize}
        \item Сообщения о загрузке модулей.
        \item Результаты подключения к vCenter.
        \item Проверка наличия и состояния vSwitch.
        \item Результаты создания или изменения параметров.
        \item Копирование порт-групп и сохранение их параметров.
    \end{itemize}
    
    \item \textbf{Диалоговые запросы.}
    \begin{itemize}
        \item Создать ли новый vSwitch?
        \item Изменить ли значение MTU?
        \item Скопировать ли порт-группы?
    \end{itemize}
    
    \item \textbf{Логирование (при необходимости).}
    \begin{itemize}
        \item Запись событий в Windows Event Viewer.
        \item Сохранение лога в текстовый файл на диск.
        \item Интеграция с системами мониторинга (Zabbix, ELK Stack и т.д.).
    \end{itemize}
\end{enumerate}

\subsubsection{Форматы данных}
\begin{itemize}
    \item \textbf{Формат входных данных:}
    \begin{itemize}
        \item Все данные передаются в виде строковых значений.
        \item Используется стандартный ввод через Read-Host.
        \item Данные о vCenter и домене хранятся в виде хэш-таблицы [Ordered]@\{\}.
    \end{itemize}
    
    \item \textbf{Формат выходных данных:}
    \begin{itemize}
        \item Текстовые сообщения в консоли PowerShell.
        \item Ответы пользователя — строковые значения (Y, y, N, n).
        \item Результаты работы возвращаются в виде объектов PowerCLI.
    \end{itemize}
\end{itemize}

\subsubsection{Примеры обработки данных}
\begin{lstlisting}[language=PowerShell, basicstyle=\small\ttfamily]
Выберите vCenter из списка:
1 - vc1.orange-ftgroup.ru
2 - vc2.orange-ftgroup.ru
Введите номер vCenter: 1
Введите имя целевого хоста (без домена): esxi-new
Введите имя исходного хоста (без домена): esxi-old
Введите имя vSwitch (например, vSwitch0, vSwitch1, vSwitch2, vSwitch3) или введите 'q' для выхода: vSwitch0
vSwitch vSwitch0 не существует на хосте esxi-new.orange-ftgroup.ru. Хотите создать его с MTU=9000? (Y/N): Y
\end{lstlisting}

\begin{lstlisting}[language=PowerShell, basicstyle=\small\ttfamily, caption=Пример вывода]
vSwitch vSwitch0 создан с MTU=9000.
Хотите скопировать настройки VLAN с esxi-old.orange-ftgroup.ru на esxi-new.orange-ftgroup.ru ? (Y/N): Y
Порт-группа PG_VLAN_10 создана.
Порт-группа PG_VLAN_20 создана.
Скрипт завершен.
\end{lstlisting}

\subsection{Требования к техническим и программным средствам}

\subsubsection{Общие положения}
Для корректного выполнения программы <<Программа конфигурации хостов виртуализации>> (шифр ЛАБ 1) необходимо соблюдение определённых требований к аппаратным и программным средствам, как на стороне клиента (рабочей станции), так и со стороны серверной инфраструктуры.

Программа предназначена для работы в интерактивном режиме через командную строку PowerShell 7 и взаимодействует с серверами VMware vCenter и ESXi-хостами.

\subsubsection{Требования к аппаратным средствам (на стороне клиента)}
\begin{itemize}
    \item \textbf{Процессор:} 1 ГГц и выше.
    \item \textbf{Оперативная память:} 2 ГБ.
    \item \textbf{Свободное место на диске:} Не менее 500 МБ.
    \item \textbf{Сетевой адаптер:} Для подключения к серверам vCenter по протоколу HTTPS.
\end{itemize}

\textbf{Примечание:} Указанные требования являются минимальными и достаточны для запуска PowerShell и модуля VMware PowerCLI. При использовании программы в составе комплексных решений или автоматизированных систем могут потребоваться более высокие характеристики.

\subsubsection{Требования к программному обеспечению (на стороне клиента)}
\begin{itemize}
    \item \textbf{Операционная система:} Windows 10 / Windows 11 / Windows Server 2016 и выше.
    \item \textbf{Интерпретатор PowerShell:} PowerShell 7.x (Core) — обязательное условие!
    \item \textbf{Установленный модуль:} VMware.PowerCLI версии не ниже 12.0.
    \item \textbf{Уровень доступа:} Пользователь должен обладать правами, позволяющими устанавливать модули PowerShell и подключаться к серверам vCenter.
    \item \textbf{Поддержка TLS:} Настройки безопасности должны позволять использование протоколов TLS 1.2 и выше.
\end{itemize}

Скрипт не поддерживается в классической версии Windows PowerShell 5.1, так как использует функции, доступные только в PowerShell 7.

\subsubsection{Требования к серверной инфраструктуре}
\begin{enumerate}
    \item \textbf{Серверы vCenter:}
    \begin{itemize}
        \item Версия vCenter Server: не ниже 7.0.
        \item Доступность по протоколу HTTPS.
        \item Наличие разрешений на чтение/запись параметров хостов у пользователя, под которым происходит подключение.
    \end{itemize}
    
    \item \textbf{ESXi-хосты:}
    \begin{itemize}
        \item Версия ESXi: не ниже 7.0.
        \item Режим обслуживания (maintenance mode) должен быть включён перед началом выполнения скрипта.
        \item Возможность создания и настройки виртуальных коммутаторов и порт-групп.
    \end{itemize}
\end{enumerate}

\subsubsection{Требования к сетевой среде}
\begin{itemize}
    \item \textbf{Доступ к серверам vCenter:} Должен быть обеспечен сетевой доступ между клиентской машиной и сервером vCenter.
    \item \textbf{Протокол передачи данных:} HTTPS (порт 443).
    \item \textbf{Брандмауэр:} Должен пропускать трафик на порт 443 для vCenter и порты, необходимые для работы ESXi.
    \item \textbf{DNS:} Корректная настройка DNS для разрешения имён серверов vCenter и ESXi-хостов.
\end{itemize}

\subsubsection{Требования к правам доступа}
Пользователь, от имени которого выполняется программа, должен иметь следующие права:
\begin{itemize}
    \item Права на подключение к серверу vCenter.
    \item Права на чтение параметров хостов.
    \item Права на создание и изменение виртуальных коммутаторов и порт-групп.
    \item Права на установку значения MTU и изменение политик безопасности.
\end{itemize}

\subsection{Распределение носителей данных}

\subsubsection{Общие положения}
Программа <<Программа конфигурации хостов виртуализации>> (шифр ЛАБ 1) представляет собой PowerShell-скрипт, не требующий установки в систему и не использующий отдельных файлов данных. Все данные, необходимые для работы программы, хранятся в оперативной памяти во время выполнения скрипта.

Однако для корректного функционирования и долгосрочного хранения программа взаимодействует с различными типами носителей данных на клиентской и серверной сторонах.

\subsubsection{Файловая структура хранения}
\begin{enumerate}
    \item \textbf{Место расположения скрипта.}
    \begin{itemize}
        \item На локальном диске рабочей станции.
        \item На сетевом диске или общем ресурсе, доступном пользователю.
        \item В репозитории системы контроля версий (например, Git).
    \end{itemize}
    
    \begin{lstlisting}[language=PowerShell, basicstyle=\small\ttfamily, caption=Пример пути хранения]
C:\Scripts\Lab1_ESXi_Configuration.ps1
    \end{lstlisting}
    
    \item \textbf{Временные файлы.}
    \begin{itemize}
        \item Программа не создаёт временных файлов на диске.
        \item Все промежуточные данные обрабатываются в оперативной памяти PowerShell.
    \end{itemize}
\end{enumerate}

\subsubsection{Использование памяти}
\begin{enumerate}
    \item \textbf{Оперативная память.}
    \begin{itemize}
        \item Содержимое модуля VMware.PowerCLI.
        \item Данные о состоянии vCenter, ESXi-хостах, виртуальных коммутаторах и порт-группах.
        \item Пользовательские параметры (выбранные хосты, vSwitch'и и т.д.).
        \item Объем используемой памяти: от 50 до 150 МБ.
    \end{itemize}
    
    \item \textbf{Кэширование.}
    \begin{itemize}
        \item PowerShell автоматически кэширует подключённые модули.
        \item Ускоряет повторный запуск программы.
    \end{itemize}
\end{enumerate}

\subsubsection{Хранение данных на стороне сервера}
\begin{enumerate}
    \item \textbf{Информация о виртуальной сети.}
    \begin{itemize}
        \item В конфигурационных файлах ESXi-хоста.
        \item В базе данных vCenter (при наличии).
        \item В виде внутренних объектов API vSphere.
    \end{itemize}
    
    \item \textbf{Резервное копирование конфигурации.}
    \begin{itemize}
        \item Рекомендуется выполнять через esxcli или vSphere Client.
    \end{itemize}
\end{enumerate}

\subsubsection{Передача данных по сети}
\begin{enumerate}
    \item \textbf{Подключение к vCenter/ESXi.}
    \begin{itemize}
        \item Протокол HTTPS (порт 443).
        \item Шифрование TLS 1.2 и выше.
    \end{itemize}
    
    \item \textbf{Объём передаваемых данных.}
    \begin{itemize}
        \item Получение информации о хосте: 10-100 КБ.
        \item Список порт-групп: 5-50 КБ на vSwitch.
        \item Общий объём трафика: 50-500 КБ за сеанс.
    \end{itemize}
\end{enumerate}

\subsubsection{Рекомендации по управлению данными}
\begin{itemize}
    \item Использовать систему контроля версий (Git) для хранения скрипта.
    \item Реализовать логгирование в текстовый файл или Windows Event Viewer.
    \item Регулярно проверять актуальность данных в разных средах vCenter.
    \item Создавать резервные копии конфигурации перед критическими операциями.
\end{itemize}

\newpage

\section{Ожидаемые технико-экономические показатели}

\subsection{Общие положения}
Программа <<Программа конфигурации хостов виртуализации>> (шифр ЛАБ 1) направлена на автоматизацию процесса настройки сетевой инфраструктуры при добавлении новых или восстановлении существующих хостов VMware ESXi.

Внедрение программы позволяет:
\begin{itemize}
    \item Повысить эффективность административных операций
    \item Сократить время на рутинные задачи
    \item Минимизировать риск ошибок, связанных с ручной настройкой параметров
\end{itemize}

\subsection{Технические показатели}
\begin{table}[h]
\centering
\caption{Основные технические характеристики}
\begin{tabular}{|p{7cm}|p{8cm}|}
\hline
\textbf{Параметр} & \textbf{Значение} \\ \hline
Среда выполнения & PowerShell 7.x \\ \hline
Поддерживаемые версии vCenter & 7.0 и выше \\ \hline
Поддерживаемые версии ESXi & 7.0 и выше \\ \hline
Требования к клиентской машине & Windows 10/11/Server, PowerShell 7, модуль VMware.PowerCLI \\ \hline
Время настройки vSwitch & 5-30 секунд (зависит от количества порт-групп) \\ \hline
Совместимость & Только PowerShell 7 (не поддерживает Windows PowerShell 5.1) \\ \hline
Тип взаимодействия & Интерактивный (через командную строку) \\ \hline
\end{tabular}
\end{table}

\subsection{Экономические показатели}
\subsubsection{Снижение трудозатрат}
\begin{itemize}
    \item Ручная настройка: 10-20 минут на хост
    \item Автоматизированная настройка: 2-5 минут на хост
    \item Экономия времени: до 75\% на каждой операции
\end{itemize}

\subsubsection{Повышение качества настройки}
Программа обеспечивает:
\begin{itemize}
    \item Единообразие конфигурации для всех хостов
    \item Снижение ошибок настройки на 80-90\%
    \item Автоматическую проверку режима обслуживания
\end{itemize}

\subsubsection{Экономический эффект}
При еженедельном использовании для 5 хостов:
\begin{itemize}
    \item Годовая экономия времени: до 60 часов
    \item Снижение простоев при восстановлении на 40-50\%
    \item Уменьшение затрат на устранение ошибок настройки
\end{itemize}

\begin{table}[h]
\centering
\caption{Сравнительные показатели эффективности}
\begin{tabular}{|l|c|c|}
\hline
\textbf{Параметр} & \textbf{Ручная настройка} & \textbf{Автоматизированная} \\ \hline
Среднее время на хост & 15 мин & 3 мин \\ \hline
Вероятность ошибки & 15-20\% & 1-2\% \\ \hline
Требуемая квалификация & Высокая & Средняя \\ \hline
Масштабируемость & Ограниченная & Высокая \\ \hline
\end{tabular}
\end{table}

\subsection{Перспективы внедрения}
\begin{itemize}
    \item Интеграция в CI/CD пайплайны
    \item Использование в регулярных сценариях тестирования
    \item Возможность расширения функционала для управления другими параметрами хостов
\end{itemize}