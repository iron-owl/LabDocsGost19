%\documentclass[12pt,a4paper]{article}
%\usepackage{tech_doc_rus}
%\makenomenclature %Закомментировать, если не нужен

%\begin{document}

%\tableofcontents

%\newpage
%\addcontentsline{toc}{section}{\tocsecindent{Обозначения и сокращения}}
%\renewcommand{\nomname}{Обозначения и сокращения}
%\printnomenclature
%\newpage

%\section*{Аннотация}


\makenomenclature

\nomenclature{ПО}{Программное обеспечение --- Совокупность программ, используемых для выполнения определённых задач на вычислительной системе}
\nomenclature{ESXi}{VMware ESXi --- Гипервизор, обеспечивающий виртуализацию на уровне ядра операционной системы}
\nomenclature{vCenter}{vCenter Server --- Центральный сервер управления инфраструктурой виртуализации от VMware}
\nomenclature{vSwitch}{Virtual Switch --- Виртуальный коммутатор, обеспечивающий сетевое взаимодействие между виртуальными машинами и физической сетью}
\nomenclature{VLAN ID}{Virtual Local Area Network Identifier --- Идентификатор виртуальной локальной сети, используемый для разделения трафика внутри одной физической сети}
\nomenclature{MTU}{Maximum Transmission Unit --- Максимальный размер передаваемого сетевого пакета, измеряется в байтах}
\nomenclature{PowerCLI}{VMware PowerCLI --- Набор PowerShell-модулей для управления инфраструктурой VMware через скрипты}
\nomenclature{PowerShell 7}{PowerShell Core 7 --- Кроссплатформенная версия PowerShell, поддерживающая современные функции и .NET Core}
\nomenclature{Port Group}{Порт-группа --- Логическая группа портов на виртуальном коммутаторе, имеющая общие параметры VLAN и безопасности}
\nomenclature{Maintenance Mode}{Режим обслуживания --- Состояние хоста ESXi, при котором он временно исключается из активной эксплуатации для проведения работ}
\nomenclature{TLS}{Transport Layer Security --- Протокол шифрования, обеспечивающий безопасную передачу данных по сети}
\nomenclature{HTTPS}{HyperText Transfer Protocol Secure --- Безопасный протокол передачи данных поверх TLS}
\nomenclature{CI/CD}{Continuous Integration / Continuous Delivery --- Методологии автоматизации разработки и доставки программного обеспечения}
\nomenclature{Git}{Git --- Система контроля версий, используемая для отслеживания изменений в коде программы}

\newpage\annotation

Настоящий документ представляет собой \textbf{Программу и методику испытаний (ПМИ)} программного средства, предназначенного для автоматизированной настройки сетевой инфраструктуры хостов VMware ESXi посредством PowerShell-скрипта с использованием модуля VMware PowerCLI. Документ разработан в соответствии с требованиями ГОСТ 19.301–79 «Единая система программной документации. Программа и методика испытаний. Требования к содержанию и оформлению».

Целью документа является определение порядка и условий проведения испытаний, подтверждающих соответствие программного средства требованиям технического задания, а также оценка его функциональных, эксплуатационных и интерфейсных характеристик.

В ПМИ изложены:

\begin{itemize}
    \item наименование и область применения объекта испытаний;
    \item цели и задачи проведения испытаний;
    \item требования, предъявляемые к программному продукту, подлежащие проверке в ходе испытаний;
    \item требования к программной документации, представляемой на испытания;
    \item состав и порядок проведения испытаний, включая перечень используемых технических и программных средств;
    \item методы испытаний, приведённые в соответствии с разделами требований;
    \item критерии оценки успешности и прекращения испытаний;
    \item формы представления результатов испытаний.
\end{itemize}

Методики испытаний включают описание процедур проверки с указанием перечней тестовых примеров, критериев оценки, ожидаемых и фактических результатов, а также сопровождены необходимыми подтверждающими материалами (логи, скриншоты, таблицы и графики), представляемыми в приложениях к документу.

Настоящая программа и методика испытаний предназначена для использования специалистами по качеству, разработчиками, системными администраторами, а также аттестующими организациями в процессе приёмки программного обеспечения, его опытной эксплуатации и внедрения в промышленную среду.

\newpage
\section{Объект испытаний}

Объектом испытаний является программное средство в виде PowerShell-скрипта, предназначенного для автоматизированной настройки сетевой конфигурации на хостах VMware ESXi, подключённых к инфраструктуре vCenter. Программа использует модуль VMware PowerCLI для выполнения операций через API vSphere.

\textbf{Наименование программы:} \textit{Скрипт автоматизированной настройки сетевой конфигурации ESXi-хостов}.

\textbf{Область применения:} программа предназначена для использования администраторами виртуальной инфраструктуры в организациях, использующих платформу VMware vSphere, с целью упрощения и ускорения рутинных операций, связанных с конфигурацией виртуальных коммутаторов (vSwitch), созданием и копированием порт-групп, а также проверки состояния хостов (например, перевода в режим обслуживания). Скрипт применяется в тестовых и рабочих средах при условии наличия доступа к vCenter и соответствующих прав.

\textbf{Обозначение программы:} \texttt{vmware\_netconfig.ps1} (или другое идентификаторное имя, присвоенное программе согласно ТЗ или внутренней системе идентификации разработчика).

\newpage
\section{Цели испытаний}

Целью проведения испытаний является подтверждение соответствия программного средства требованиям технического задания и нормам, установленным в нормативных документах, а также оценка его функциональной полноты, надёжности и удобства эксплуатации.

\medskip

Основные задачи испытаний:

\begin{enumerate}
    \item Проверка корректности выполнения ключевых функций программы: подключение к vCenter, выбор и проверка состояния хостов, создание и изменение виртуальных коммутаторов (vSwitch), копирование порт-групп.
    \item Оценка устойчивости программы к ошибкам ввода и отсутствию необходимых зависимостей (например, модуля VMware.PowerCLI).
    \item Проверка соответствия интерфейса программы требованиям удобства использования, полноты и корректности диалоговых сообщений.
    \item Верификация совместимости с целевыми версиями vCenter, ESXi и PowerShell.
    \item Подтверждение выполнения требований безопасности и защиты информации в процессе эксплуатации программы.
\end{enumerate}

Результаты испытаний служат основанием для принятия решения о пригодности программного средства к внедрению и эксплуатации в заданной информационной среде.

\newpage
\section{Цели испытаний}

В ходе испытаний проверяются следующие требования к программному средству, сформулированные в техническом задании:

\subsection{Функциональные требования}

\begin{itemize}
    \item Возможность подключения к указанному серверу vCenter с использованием учетных данных пользователя.
    \item Корректный выбор исходного и целевого ESXi-хостов из списка доступных.
    \item Проверка режима обслуживания целевого хоста; запрет выполнения операций, если хост не находится в режиме обслуживания.
    \item Возможность создания нового виртуального коммутатора (vSwitch) с заданным MTU (9000).
    \item Возможность изменения параметров существующего vSwitch, включая настройку MTU.
    \item Копирование порт-групп с одного хоста на другой с сохранением настроек VLAN и параметров безопасности.
    \item Организация интерактивного диалога с пользователем, включая запросы и обработку ответов.
\end{itemize}

\subsection{Требования к надёжности}

\begin{itemize}
    \item Обработка ошибок подключения к vCenter и ESXi-хостам с корректным информированием пользователя.
    \item Устойчивость к некорректным данным, вводимым пользователем.
    \item Обработка отсутствия необходимых модулей (например, VMware.PowerCLI) и автоматическая их установка при необходимости.
\end{itemize}

\subsection{Требования к интерфейсу}

\begin{itemize}
    \item Использование русского языка в диалогах и сообщениях.
    \item Четкое форматирование выводимых сообщений и запросов.
    \item Возможность интерактивного выбора действий пользователем.
\end{itemize}

\subsection{Требования к совместимости}

\begin{itemize}
    \item Совместимость с версиями VMware vCenter и ESXi, поддерживающими PowerCLI.
    \item Поддержка версий PowerShell не ниже 7.x.
    \item Корректное взаимодействие с API vSphere через VMware.PowerCLI.
\end{itemize}

\subsection{Требования к безопасности}

\begin{itemize}
    \item Обеспечение безопасного подключения к vCenter через защищённые протоколы (HTTPS, TLS).
    \item Отсутствие обработки и хранения персональных данных.
    \item Проверка прав доступа пользователя к vCenter для выполнения операций.
\end{itemize}

\subsection{Требования к сохранению данных}

\begin{itemize}
    \item Временное хранение параметров и настроек в оперативной памяти во время работы программы.
    \item Возможность ведения логирования при расширении функционала (опционально).
\end{itemize}

\newpage
\section{Требования к программной документации}

Для проведения испытаний предъявляется полный комплект программной документации, соответствующий требованиям технического задания и стандартов ЕСКД и ЕСПД. В состав документации входят следующие основные документы:

\begin{enumerate}
    \item \textbf{Техническое задание (ТЗ)} — ЛАБ 1.ТЗ. Документ, определяющий цели, задачи, функциональные и технические требования к программе.
    \item \textbf{Пояснительная записка (ПЗ)} — ЛАБ 1.ПЗ. Документ, описывающий общие положения, обоснование разработки, принципы работы программы.
    \item \textbf{Руководство системного программиста (РСП)} — ЛАБ 1.РСП. Руководство по установке, конфигурированию и сопровождению программного продукта для системных специалистов.
    \item \textbf{Программа и методика испытаний (ПМИ)} — ЛАБ 1.ПМИ. Документ, регламентирующий порядок проведения испытаний, методы тестирования и критерии оценки результатов.
    \item \textbf{Формуляр (ФО)} — ЛАБ 1.ФО. Сводный документ с основными сведениями о программе, её версии, авторах и сроках разработки.
    \item \textbf{Ведомость эксплуатационных документов (ВЭД)} — ЛАБ 1.ВЭД. Перечень эксплуатационных документов, прилагаемых к программному продукту.
\end{enumerate}

Специальные требования к программной документации, если они заданы в техническом задании, должны быть учтены и отражены в соответствующих разделах документов. При этом основное внимание уделяется полноте, актуальности и соответствию документации фактической реализации программы.

Все документы должны быть оформлены в соответствии с установленными стандартами и нормами, обеспечивая однозначное понимание требований и возможности их проверки в ходе испытаний.

\newpage
\section{Состав и порядок испытаний}

\subsection{Технические средства испытаний}

Для проведения испытаний используется следующий набор технических средств:

\begin{itemize}
    \item Рабочая станция или сервер с установленной операционной системой Windows 10, Windows 11 или Windows Server (версия и конфигурация в соответствии с требованиями ТЗ).
    \item Сетевое оборудование и инфраструктура, обеспечивающие доступ к тестовым серверам vCenter и ESXi.
    \item Средства мониторинга и диагностики для контроля состояния оборудования и сетевого взаимодействия.
\end{itemize}

\subsection{Программные средства испытаний}

В состав программных средств входят:

\begin{itemize}
    \item PowerShell версии 7.x и выше с установленным модулем VMware.PowerCLI версии не ниже 12.0.
    \item Средства логгирования и диагностики, применяемые для сбора и анализа результатов тестирования.
    \item Тестовые сценарии и скрипты, реализующие проверку функций и требований программы.
\end{itemize}

\subsection{Порядок проведения испытаний}

Испытания проводятся по следующему алгоритму:

\begin{enumerate}
    \item Подготовка испытательной среды:
    \begin{itemize}
        \item Установка и настройка операционной системы на тестовой рабочей станции.
        \item Установка PowerShell и необходимых модулей.
        \item Настройка сетевого подключения к тестовым серверам vCenter и ESXi.
    \end{itemize}
    \item Проверка доступности и корректности работы внешних компонентов и зависимостей.
    \item Последовательное выполнение тестовых сценариев согласно методике испытаний:
    \begin{itemize}
        \item Проверка подключения к vCenter.
        \item Выбор и верификация хостов.
        \item Проверка режимов обслуживания и управления vSwitch.
        \item Тестирование создания, изменения и копирования конфигураций сетевых объектов.
        \item Проверка обработки ошибок и некорректного ввода.
    \end{itemize}
    \item Сбор и анализ результатов испытаний, формирование отчетности.
    \item В случае выявления несоответствий — регистрация ошибок и проведение повторных испытаний после внесения исправлений.
\end{enumerate}

\newpage
\section{Методы испытаний}

Методы испытаний построены в соответствии с перечнем требований, изложенных в разделах \textit{«Требования к программе»} и \textit{«Требования к программной документации»}. Каждый метод испытаний описывает способ проверки отдельного показателя с указанием целей, пошагового порядка действий, ожидаемого результата и критериев приемки.

\subsection{Методы проверки функциональных требований}

\begin{enumerate}
    \item \textbf{Проверка подключения к vCenter}
    \begin{itemize}
        \item \textit{Цель:} Убедиться в возможности успешного подключения к серверу vCenter.
        \item \textit{Методика:} Выполнить запуск тестируемой программы, указать параметры подключения (адрес, логин, пароль).
        \item \textit{Ожидаемый результат:} Программа подключается без ошибок, отображает сведения о доступных хостах.
    \end{itemize}

    \item \textbf{Выбор хостов и проверка режима обслуживания}
    \begin{itemize}
        \item \textit{Методика:} Через интерактивный ввод указать исходный и целевой хосты. Программа проверяет, находится ли хост в режиме обслуживания.
        \item \textit{Ожидаемый результат:} Корректное определение статуса хостов.
    \end{itemize}

    \item \textbf{Работа с виртуальными коммутаторами (vSwitch)}
    \begin{itemize}
        \item \textit{Методика:} Создать новый vSwitch, изменить параметры существующего (например, MTU), проверить применимость изменений.
        \item \textit{Ожидаемый результат:} Все изменения применяются корректно, без ошибок, результаты отображаются в логах.
    \end{itemize}

    \item \textbf{Копирование порт-групп}
    \begin{itemize}
        \item \textit{Методика:} Выбрать исходный vSwitch, инициировать копирование порт-групп на целевой хост.
        \item \textit{Ожидаемый результат:} Все группы успешно копируются, включая параметры VLAN, MTU, политики безопасности.
    \end{itemize}

    \item \textbf{Интерактивный диалог с пользователем}
    \begin{itemize}
        \item \textit{Методика:} Проверить корректность диалогов, наличие проверок ввода и повторных запросов при ошибках.
        \item \textit{Ожидаемый результат:} Программа обрабатывает ввод, дает пояснения и повторно запрашивает некорректные значения.
    \end{itemize}
\end{enumerate}

\subsection{Методы проверки надежности}

\begin{itemize}
    \item \textbf{Проверка устойчивости к некорректному вводу:} Ввод недопустимых значений (пустой логин, неверный IP) должен вызывать обработанные ошибки.
    \item \textbf{Отсутствие зависимостей:} Проверка реакции при отсутствии модуля PowerCLI. Ожидаемый результат — вывод понятного сообщения с рекомендацией установки.
\end{itemize}

\subsection{Методы проверки интерфейса}

\begin{itemize}
    \item \textbf{Язык сообщений:} Все взаимодействие должно вестись на русском языке.
    \item \textbf{Формат сообщений:} Проверка соответствия шаблону: чёткие и информативные сообщения, без технических стеков.
\end{itemize}

\subsection{Методы проверки информационной совместимости}

\begin{itemize}
    \item \textbf{Совместимость с разными версиями PowerShell и vSphere:} Провести испытания на различных конфигурациях.
    \item \textbf{Взаимодействие с API:} Проверка корректного ответа от vCenter через командлеты PowerCLI.
\end{itemize}

\subsection{Методы проверки требований к защите информации}

\begin{itemize}
    \item \textbf{Безопасность передачи данных:} Проверка использования HTTPS при взаимодействии с vCenter.
    \item \textbf{Ограничение прав:} Подключение от имени пользователя с ограниченными правами и проверка отклонения операций.
\end{itemize}

\subsection{Методы проверки сохранения данных}

\begin{itemize}
    \item \textbf{Временное хранение параметров:} Проверка, что параметры используются только в оперативной памяти.
    \item \textbf{Возможность логгирования:} Провести тестовый запуск с флагом логгирования (если реализовано) и проверить создание лог-файлов.
\end{itemize}

\subsection{Методы проверки программной документации}

\begin{itemize}
    \item \textbf{Проверка полноты комплекта:} Визуальная проверка наличия всех необходимых документов.
    \item \textbf{Проверка соответствия требованиям:} Сравнение содержимого каждого документа с ТЗ.
\end{itemize}

\subsection{Контрольные материалы}

К каждому испытанию прилагаются:

\begin{itemize}
    \item Перечень тестовых примеров (входные данные, параметры подключения и конфигурации).
    \item Скриншоты или лог-файлы, подтверждающие выполнение операций.
    \item Таблицы с ожидаемыми и фактическими результатами.
    \item Описание выявленных отклонений (при наличии).
\end{itemize}






