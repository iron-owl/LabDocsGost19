%\documentclass[12pt,a4paper]{article}
%\usepackage{tech_doc_rus}
%\makenomenclature %Закомментировать, если не нужен

%\begin{document}

%\tableofcontents

%\newpage
%\addcontentsline{toc}{section}{\tocsecindent{Обозначения и сокращения}}
%\renewcommand{\nomname}{Обозначения и сокращения}
%\printnomenclature
%\newpage

%\section*{Аннотация}


\makenomenclature

\nomenclature{ПО}{Программное обеспечение --- Совокупность программ, используемых для выполнения определённых задач на вычислительной системе}
\nomenclature{ESXi}{VMware ESXi --- Гипервизор, обеспечивающий виртуализацию на уровне ядра операционной системы}
\nomenclature{vCenter}{vCenter Server --- Центральный сервер управления инфраструктурой виртуализации от VMware}
\nomenclature{vSwitch}{Virtual Switch --- Виртуальный коммутатор, обеспечивающий сетевое взаимодействие между виртуальными машинами и физической сетью}
\nomenclature{VLAN ID}{Virtual Local Area Network Identifier --- Идентификатор виртуальной локальной сети, используемый для разделения трафика внутри одной физической сети}
\nomenclature{MTU}{Maximum Transmission Unit --- Максимальный размер передаваемого сетевого пакета, измеряется в байтах}
\nomenclature{PowerCLI}{VMware PowerCLI --- Набор PowerShell-модулей для управления инфраструктурой VMware через скрипты}
\nomenclature{PowerShell 7}{PowerShell Core 7 --- Кроссплатформенная версия PowerShell, поддерживающая современные функции и .NET Core}
\nomenclature{Port Group}{Порт-группа --- Логическая группа портов на виртуальном коммутаторе, имеющая общие параметры VLAN и безопасности}
\nomenclature{Maintenance Mode}{Режим обслуживания --- Состояние хоста ESXi, при котором он временно исключается из активной эксплуатации для проведения работ}
\nomenclature{TLS}{Transport Layer Security --- Протокол шифрования, обеспечивающий безопасную передачу данных по сети}
\nomenclature{HTTPS}{HyperText Transfer Protocol Secure --- Безопасный протокол передачи данных поверх TLS}
\nomenclature{CI/CD}{Continuous Integration / Continuous Delivery --- Методологии автоматизации разработки и доставки программного обеспечения}
\nomenclature{Git}{Git --- Система контроля версий, используемая для отслеживания изменений в коде программы}

\newpage\annotation

Настоящий документ представляет собой техническое задание на разработку программы «Программа конфигурации хостов виртуализации» (шифр ЛАБ 1), предназначенной для автоматизации процесса сетевой настройки хостов VMware ESXi при их добавлении или восстановлении в инфраструктуре внутреннего облачного окружения.

В документе изложены цели и назначение программы, определены её функциональные и эксплуатационные характеристики, указаны требования к программному и аппаратному обеспечению, описаны этапы разработки и формы представления результатов.

Документ разработан в соответствии с требованиями ГОСТ 19.201-78 и предназначен для использования при проектировании, разработке, тестировании и сопровождении программного средства. Техническое задание предназначено для руководителей проектов, системных аналитиков, разработчиков, специалистов по тестированию и технической поддержке.

\newpage
\section{Введение}

\subsection{Наименование программы}
\begin{itemize}
    \item \textbf{Полное наименование:} <<Программа конфигурации хостов виртуализации>>
    \item \textbf{Шифр проекта:} ЛАБ 1
    \item \textbf{Тип программного средства:} PowerShell-скрипт
\end{itemize}

\textbf{Назначение:} Автоматизация сетевой настройки новых или восстанавливаемых хостов VMware ESXi при их добавлении в инфраструктуру внутреннего облачного окружения.

\subsection{Наименование разработчика и заказчика}
\begin{table}[h]
\centering
\begin{tabular}{|l|l|}
\hline
\textbf{Разработчик:} & МГТУ ГА, Иванов И. И. \\ \hline
\textbf{Заказчик:} & МГТУ ГА, Аникаев  К. П. \\ \hline
\end{tabular}
\end{table}

\subsection{Основание для разработки}
\begin{itemize}
    \item \textbf{Цель разработки:}
    \begin{itemize}
        \item Стандартизация процессов настройки сетевых параметров ESXi-хостов
        \item Автоматизация рутинных операций
        \item Минимизация ошибок при ручной настройке
    \end{itemize}
\end{itemize}

\subsection{Цель создания программы}
\begin{itemize}
    \item Обеспечение унифицированной настройки виртуальных коммутаторов (vSwitch) и порт-групп
    \item Основные функции:
    \begin{itemize}
        \item Проверка наличия и параметров виртуальных коммутаторов
        \item Установка MTU=9000
        \item Создание отсутствующих vSwitch
        \item Копирование порт-групп с сохранением:
        \begin{itemize}
            \item VLAN ID
            \item Политик безопасности
        \end{itemize}
    \end{itemize}
\end{itemize}

\subsection{Источники разработки}
\begin{enumerate}
    \item Официальная документация:
    \begin{itemize}
        \item VMware vSphere API
        \item VMware PowerCLI Reference
    \end{itemize}
    \item Рекомендации VMware по сетевой настройке
    \item Внутренние стандарты компании по эксплуатации:
    \begin{itemize}
        \item Виртуальной инфраструктуры
        \item Сетевой безопасности
    \end{itemize}
\end{enumerate}

\newpage
\section{Назначение и условия применения}

\subsection{Назначение программы}
Программа <<Программа конфигурации хостов виртуализации>> (шифр ЛАБ 1) предназначена для автоматизации процесса сетевой настройки хостов VMware ESXi при их добавлении или восстановлении в инфраструктуре внутреннего облачного окружения.

\begin{itemize}
    \item \textbf{Тип программы:} PowerShell-скрипт с использованием модуля VMware.PowerCLI
    \item \textbf{Основные функции:}
    \begin{itemize}
        \item Обеспечение унифицированной конфигурации виртуальных коммутаторов (vSwitch).
        \item Настройка порт-групп с сохранением параметров.
        \item Стандартизация сетевых параметров.
    \end{itemize}
    \item \textbf{Преимущества:}
    \begin{itemize}
        \item Исключение ошибок ручной настройки.
        \item Сокращение времени конфигурации хостов на 70-80\%.
        \item Повышение надёжности виртуальной инфраструктуры.
    \end{itemize}
\end{itemize}

\subsection{Область применения}
\begin{enumerate}
    \item \textbf{Настройка новых ESXi-хостов}
    \begin{itemize}
        \item Массовое внедрение хостов.
        \item Приведение к единому сетевому стандарту.
    \end{itemize}
    
    \item \textbf{Восстановление после сбоев}
    \begin{itemize}
        \item Автоматическое восстановление конфигурации.
        \item Минимизация времени простоя.
    \end{itemize}
    
    \item \textbf{Стандартизация параметров}
    \begin{itemize}
        \item Единые настройки MTU (9000).
        \item Консистентные VLAN ID и политики безопасности.
    \end{itemize}
    
    \item \textbf{Интеграция в автоматизацию}
    \begin{itemize}
        \item CI/CD пайплайны.
        \item Системы оркестрации.
    \end{itemize}
\end{enumerate}

\subsection{Условия эксплуатации}
\subsubsection{Техническая среда}
\begin{table}[h]
\centering
\begin{tabular}{|l|l|}
\hline
\textbf{Компонент} & \textbf{Требования} \\ \hline
ОС & Windows 10/11/Server 2016+ \\ \hline
PowerShell & Версия 7.x (Core) \\ \hline
Сетевое подключение & HTTPS (порт 443) \\ \hline
Дополнительно & Доступ в интернет для установки модулей \\ \hline
\end{tabular}
\end{table}

\subsubsection{Требования к VMware}
\begin{itemize}
    \item vCenter Server 7.0+.
    \item ESXi 7.0+.
    \item Режим обслуживания хоста.
    \item Эталонная конфигурация на исходном хосте.
\end{itemize}

\subsubsection{Права доступа}
\begin{itemize}
    \item Подключение к vCenter.
    \item Чтение/изменение сетевой конфигурации.
    \item Установка модулей PowerShell (при необходимости).
\end{itemize}

\subsubsection{Ограничения}
\begin{itemize}
    \item Только PowerShell 7 (не поддерживается Windows PowerShell 5.1).
    \item Требуется стабильное соединение с vCenter.
    \item Обязательная проверка прав доступа.
    \item Не рекомендуется использовать вне режима обслуживания.
\end{itemize}

\section{Требования к программе}

\subsection{Требования к функциональным характеристикам}
Программа должна обеспечивать выполнение следующих функций:

\begin{enumerate}
    \item \textbf{Поддержка выбора vCenter}
    \begin{itemize}
        \item Программа предоставляет пользователю список доступных серверов vCenter.
        \item Выбор производится по номеру из предложенного списка.
        \item При некорректном вводе программа выводит сообщение об ошибке и повторяет запрос.
    \end{itemize}
    
    \item \textbf{Подключение к серверу vCenter}
    \begin{itemize}
        \item Программа подключается к выбранному серверу vCenter.
        \item Перед подключением отключается проверка сертификатов для упрощения взаимодействия.
        \item Пользователь вводит учетные данные для аутентификации.
    \end{itemize}
    
    \item \textbf{Ввод данных о хостах}
    \begin{itemize}
        \item Пользователь указывает имя целевого хоста (без домена).
        \item Пользователь указывает имя исходного хоста (без домена).
        \item К имени каждого хоста автоматически добавляется домен, указанный в скрипте.
    \end{itemize}
    
    \item \textbf{Проверка режима обслуживания}
    \begin{itemize}
        \item Программа проверяет, находится ли целевой хост в режиме обслуживания (maintenance mode).
        \item Если хост не находится в нужном состоянии, выполнение программы прекращается.
    \end{itemize}
    
    \item \textbf{Интерактивный выбор vSwitch}
    \begin{itemize}
        \item Пользователь вводит имя виртуального коммутатора (например, vSwitch0, vSwitch1).
        \item Допускается повторный выбор нескольких vSwitch'ей за один запуск программы.
        \item Возможность выхода из цикла ввода символов q или Q.
    \end{itemize}
    
    \item \textbf{Проверка наличия и MTU vSwitch}
    \begin{itemize}
        \item Программа проверяет наличие указанного vSwitch на целевом хосте.
        \item Если vSwitch существует, программа проверяет значение MTU:
        \begin{itemize}
            \item если MTU = 9000 → вывод соответствующего сообщения;
            \item если MTU ≠ 9000 → предлагается изменить значение на 9000;
            \item если vSwitch отсутствует → предлагается создать его с MTU = 9000.
        \end{itemize}
    \end{itemize}
    
    \item \textbf{Копирование порт-групп}
    \begin{itemize}
        \item По согласию пользователя программа копирует порт-группы с исходного хоста на целевой.
        \item Сохраняются параметры:
        \begin{itemize}
            \item VLAN ID;
            \item политики безопасности (AllowPromiscuous, MacChanges, ForgedTransmits).
        \end{itemize}
    \end{itemize}
    
    \item \textbf{Завершение работы}
    \begin{itemize}
        \item После окончания всех операций программа отключается от сервера vCenter.
        \item Выводится сообщение о завершении работы.
    \end{itemize}
\end{enumerate}

\subsection{Требования к надежности}
Программа должна быть устойчивой к типовым ошибкам и обеспечивать корректное поведение при следующих ситуациях:

\begin{itemize}
    \item \textbf{Некорректный ввод пользователя} - программа должна игнорировать неверные значения и запрашивать ввод повторно.
    \item \textbf{Отказ подключения к vCenter} - программа должна вывести сообщение об ошибке и завершить работу без аварийного прерывания.
    \item \textbf{Хост не в режиме обслуживания} - программа должна выдать предупреждение и завершить выполнение.
    \item \textbf{Отсутствие модуля VMware.PowerCLI} - программа должна автоматически установить модуль или выдать понятное сообщение об ошибке.
    \item \textbf{Сбой при копировании порт-групп} - программа должна продолжить выполнение, сохранив информацию о возникшей ошибке.
\end{itemize}

\subsection{Требования к интерфейсу пользователя}
Программа реализована как консольное приложение, предназначенное для запуска в PowerShell 7.

\begin{itemize}
    \item \textbf{Основные требования:}
    \begin{itemize}
        \item Все действия выполняются через командную строку.
        \item Сообщения выводятся на русском языке.
        \item Вывод информации должен быть структурированным и понятным.
        \item Все действия требуют подтверждения пользователя перед выполнением.
    \end{itemize}
    
    \item \textbf{Формат вывода:}
    \begin{itemize}
        \item Информационные сообщения должны начинаться с префикса:
        \begin{lstlisting}
"vSwitch vSwitch0 найден с MTU=9000."
        \end{lstlisting}
        \item Диалоговые запросы должны содержать варианты ответа:
        \begin{lstlisting}
"Хотите создать vSwitch vSwitch0 с MTU=9000? (Y/N):"
        \end{lstlisting}
    \end{itemize}
    
    \item \textbf{Требования к локализации:}
    \begin{itemize}
        \item Язык интерфейса: русский.
        \item Сообщения об ошибках также должны быть на русском языке.
    \end{itemize}
    
    \item \textbf{Требования к интерактивности:}
    \begin{itemize}
        \item Программа должна предоставлять возможность выбора действий.
        \item Все ключевые операции должны подтверждаться пользователем.
        \item Должна быть предусмотрена возможность досрочного выхода.
    \end{itemize}
\end{itemize}

\subsection{Требования к составу и параметрам технических средств}
\begin{table}[h]
\centering
\begin{tabular}{|p{7cm}|p{8cm}|}
\hline
\textbf{Компонент} & \textbf{Требования} \\ \hline
\multicolumn{2}{|p{15cm}|}{\textbf{Аппаратные требования (клиент)}} \\ \hline
Процессор & 1 ГГц и выше \\ \hline
Оперативная память & 2 ГБ \\ \hline
Место на диске & Не менее 500 МБ свободного места \\ \hline
Сетевой адаптер & Для подключения к серверам vCenter по HTTPS \\ \hline
\multicolumn{2}{|l|}{\textbf{Программные требования}} \\ \hline
ОС & Windows 10 / 11 / Server 2016 и выше \\ \hline
PowerShell & PowerShell 7.x (Core) \\ \hline
Установленный модуль & VMware.PowerCLI версии не ниже 12.0 \\ \hline
Уровень доступа & Права на установку модулей и подключение к vCenter \\ \hline
Протокол связи & TLS 1.2 и выше \\ \hline
Брандмауэр & Должен пропускать трафик на порт 443 (HTTPS) \\ \hline
\multicolumn{2}{|l|}{\textbf{Серверная инфраструктура}} \\ \hline
Версия vCenter & 7.0 и выше \\ \hline
Версия ESXi & 7.0 и выше \\ \hline
Режим обслуживания & Целевой хост должен быть в maintenance mode \\ \hline
Доступность & Сервер vCenter и ESXi-хосты должны быть доступны по сети \\ \hline
\end{tabular}
\end{table}

\subsection{Требования к информационной и программной совместимости}
Программа должна быть совместима со следующими компонентами:

\begin{itemize}
    \item Серверы vCenter версий 7.0 и выше.
    \item ESXi-хосты версий 7.0 и выше.
    \item PowerShell 7.x (Core), несовместима с Windows PowerShell 5.1.
    \item Системы контроля версий (Git, SVN и др.) - при необходимости хранения скрипта в репозитории.
\end{itemize}

Программа должна корректно работать:

\begin{itemize}
    \item На различных Windows-платформах (рабочие станции, серверы).
    \item В составе автоматизированных сценариев и пайплайнов DevOps.
    \item Без дополнительных прав, кроме стандартных для PowerShell.
\end{itemize}

\subsection{Требования к защите информации}
Программа не содержит механизмов шифрования данных и не обрабатывает персональные данные. Однако она взаимодействует с серверами vCenter, поэтому должны соблюдаться следующие меры безопасности:

\begin{itemize}
    \item Все сетевые взаимодействия должны происходить по протоколу HTTPS (порт 443).
    \item Использование только проверенных и доверенных источников установки модулей PowerShell.
    \item Обеспечение ограничений прав доступа на уровне vCenter (минимально необходимые привилегии).
    \item Запрет на выполнение скрипта в небезопасной среде (например, на публичных терминалах).
\end{itemize}

\subsection{Требования к надежности сохранения данных}
Все данные, используемые программой, являются временными и хранятся в оперативной памяти во время выполнения скрипта. Программа не записывает данные на диск, но может быть расширена следующим образом:

\begin{table}[h]
\centering
\begin{tabular}{|p{5cm}|p{5cm}|p{5cm}|}
\hline
\textbf{Тип данных} & \textbf{Место хранения} & \textbf{Срок хранения} \\ \hline
Конфигурация vSwitch & Временные данные в памяти & До завершения работы программы \\ \hline
Логи выполнения & Текстовый файл или Windows Event Viewer & Определяется политикой предприятия \\ \hline
Результаты копирования порт-групп & Сохраняются на ESXi-хосте через API vSphere & Бессрочно, до изменения конфигурации \\ \hline
\end{tabular}
\end{table}

Программа должна обеспечивать:

\begin{itemize}
    \item Восстановление после временных сбоев.
    \item Сохранение состояния при внезапном прерывании (при использовании в расширенных версиях).
    \item Возможность повторного запуска без потери данных.
\end{itemize}

\section{Стадии и этапы разработки}

\begin{itemize}
    \item Разработка программы «Программа конфигурации хостов виртуализации» (шифр ЛАБ 1) осуществляется в соответствии с установленными стадиями и этапами, предусмотренными ГОСТ 19.101-77 — «Единая система программной документации. Стадии разработки программ и программных изделий».
    
    \item Все работы по созданию программы проводятся в несколько стадий, начиная с технического предложения и заканчивая опытной эксплуатацией. Каждая стадия включает в себя определённый перечень работ и результатов, подлежащих оформлению в виде документов или отчётов.
\end{itemize}

\subsection{Предпроектные работы (техническое предложение)}

\textbf{Основные задачи:}
\begin{enumerate}
    \item Формирование исходных требований к программе.
    \item Анализ существующих решений и выявление потребности в новом ПО.
    \item Оценка целесообразности разработки.
\end{enumerate}

\textbf{Результат:}
\begin{itemize}
    \item Техническое предложение (необязательно при внутреннем заказе).
    \item Примерное описание функциональности программы.
    \item Возможные варианты реализации и используемые технологии.
\end{itemize}

\textbf{Участники:}
\begin{itemize}
    \item Заказчик
    \item Разработчик
    \item Системный аналитик
\end{itemize}

\subsection{Техническое задание (ТЗ)}

\textbf{Основные задачи:}
\begin{enumerate}
    \item Детализация требований к программе.
    \item Определение назначения, условий применения и состава программы.
    \item Установление технических характеристик и требований к надежности, совместимости, защите информации и т.д.
\end{enumerate}

\textbf{Результат:}
\begin{itemize}
    \item Подготовленный и утверждённый документ Техническое задание (ТЗ) по ГОСТ 19.201-78.
    \item Согласование ТЗ с заказчиком и утверждение.
\end{itemize}

\textbf{Участники:}
\begin{itemize}
    \item Заказчик
    \item Разработчик
    \item Отдел качества / технический контроль
\end{itemize}

\subsection{Эскизный проект}

\textbf{Основные задачи:}
\begin{enumerate}
    \item Выбор архитектуры и основных принципов реализации.
    \item Определение структуры скрипта.
    \item Разработка алгоритмов взаимодействия с vCenter и ESXi.
    \item Выбор технологий (PowerShell 7, VMware PowerCLI).
\end{enumerate}

\textbf{Результат:}
\begin{itemize}
    \item Пояснительная записка (ПЗ), включающая описание архитектуры и логики работы программы.
    \item Алгоритмы выполнения ключевых операций.
    \item Описание используемых командлетов и API.
\end{itemize}

\textbf{Участники:}
\begin{itemize}
    \item Системный программист
    \item DevOps-инженер
    \item Архитектор решений
\end{itemize}

\subsection{Технический проект}

\textbf{Основные задачи:}
\begin{enumerate}
    \item Детальная проработка функциональности программы.
    \item Создание прототипа или демонстрационной версии.
    \item Подготовка спецификаций на модули и компоненты.
    \item Написание первичной документации.
\end{enumerate}

\textbf{Результат:}
\begin{itemize}
    \item Частично реализованный прототип программы.
    \item Спецификация модулей и функций.
    \item Документы: пояснительная записка (ПЗ), руководство системного программиста (РСП), инструкция по установке (ИН).
\end{itemize}

\textbf{Участники:}
\begin{itemize}
    \item Разработчик PowerShell-скрипта;
    \item Специалист по тестированию;
    \item Автор технической документации.
\end{itemize}

\subsection{Рабочее проектирование и разработка}

\textbf{Основные задачи:}
\begin{enumerate}
    \item Реализация программы согласно техническому проекту.
    \item Настройка взаимодействия с vCenter и ESXi.
    \item Внедрение всех функциональных возможностей.
    \item Проведение внутреннего тестирования.
\end{enumerate}

\textbf{Результат:}
\begin{itemize}
    \item Рабочая версия программы.
    \item Исполняемый файл (скрипт) с подробными комментариями.
    \item Обновлённая документация (РО, РСП, ИН).
    \item Протоколы внутренних испытаний.
\end{itemize}

\textbf{Участники:}
\begin{itemize}
    \item Разработчик (автор скрипта);
    \item Системный администратор (тестирование);
    \item QA-инженер (проверка корректности работы).
\end{itemize}

\subsection{Внедрение и опытная эксплуатация}

\textbf{Основные задачи:}
\begin{enumerate}
    \item Проведение внедрения программы в тестовой среде.
    \item Проверка работы программы на реальных ESXi-хостах.
    \item Сбор обратной связи от пользователей.
    \item Корректировка программы и документации.
\end{enumerate}

\textbf{Результат:}
\begin{itemize}
    \item Версия программы, готовая к использованию.
    \item Отчет об опытной эксплуатации.
    \item Обновлённые документы с учётом изменений.
    \item Акт внедрения программы.
\end{itemize}

\textbf{Участники:}
\begin{itemize}
    \item Системные администраторы;
    \item DevOps-инженеры;
    \item Заказчик;
    \item Ответственный за сопровождение.
\end{itemize}

\subsection{Сопровождение и развитие}

\textbf{Основные задачи:}
\begin{enumerate}
    \item Устранение выявленных ошибок и неточностей.
    \item Внесение изменений по запросам пользователей.
    \item Адаптация программы под новые версии vCenter и ESXi.
    \item Расширение функционала при необходимости.
\end{enumerate}

\textbf{Результат:}
\begin{itemize}
    \item Обновлённая версия программы.
    \item Обновлённая документация.
    \item Журнал изменений (change log).
    \item Техническая поддержка пользователей.
\end{itemize}

\textbf{Участники:}
\begin{itemize}
    \item Разработчик;
    \item Системные администраторы;
    \item Специалисты техподдержки;
    \item Заказчик.
\end{itemize}

\subsection{Сроки выполнения этапов}

\begin{itemize}
    \item Разработка ТЗ — 1 неделя;
    \item Эскизный проект — 1–2 недели;
    \item Технический проект — 1–2 недели;
    \item Разработка и тестирование — 2–3 недели;
    \item Внедрение и опытная эксплуатация — 1–2 недели;
    \item Сопровождение — Бессрочно (по мере необходимости).
\end{itemize}

